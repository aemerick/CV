% set \longtrue to make a more detailed CV.
% set \longfalse to make a shorter CV
\newif\iflong
\longtrue

\newif\iflongexperience

\iflong
	\longexperiencetrue
\fi
%\longfalse

%%%%%%%%%%%%%%%%%%%%%%%%%%%%%%%%%%%%%%%%%%%%%%%%%%%%%%%%%%%%%%%%%%%%%%%%
%%%%%%%%%%%%%%%%%%%%%% Simple LaTeX CV Template %%%%%%%%%%%%%%%%%%%%%%%%
%%%%%%%%%%%%%%%%%%%%%%%%%%%%%%%%%%%%%%%%%%%%%%%%%%%%%%%%%%%%%%%%%%%%%%%%
\documentclass[10pt]{article}

% for Roman serrif font
%\usepackage{times}
%\renewcommand{\familydefault}{\sfdefault}
% This is a helpful package that puts math inside length specifications
\usepackage{calc}
\usepackage{comment}

% Simpler bibsection for CV sections
% (thanks to natbib for inspiration)
\makeatletter
\newlength{\bibhang}
\setlength{\bibhang}{1em} %1em}
\newlength{\bibsep}
 {\@listi \global\bibsep\itemsep \global\advance\bibsep by\parsep}
\newenvironment{bibsection}%
        {\begin{enumerate}{}{%
%        {\begin{list}{}{%
       \setlength{\leftmargin}{\bibhang}%
       \setlength{\itemindent}{-\leftmargin}%
       \setlength{\itemsep}{\bibsep}%
       \setlength{\parsep}{\z@}%
        \setlength{\partopsep}{0pt}%
        \setlength{\topsep}{0pt}}}
        {\end{enumerate}\vspace{-.6\baselineskip}}
%        {\end{list}\vspace{-.6\baselineskip}}
\makeatother

% Layout: Puts the section titles on left side of page
\reversemarginpar




%
%         PAPER SIZE, PAGE NUMBER, AND DOCUMENT LAYOUT NOTES:
%
% The next \usepackage line changes the layout for CV style section
% headings as marginal notes. It also sets up the paper size as either
% letter or A4. By default, letter was used. If A4 paper is desired,
% comment out the letterpaper lines and uncomment the a4paper lines.
%
% As you can see, the margin widths and section title widths can be
% easily adjusted.
%
% ALSO: Notice that the includefoot option can be commented OUT in order
% to put the PAGE NUMBER *IN* the bottom margin. This will make the
% effective text area larger.
%
% IF YOU WISH TO REMOVE THE ``of LASTPAGE'' next to each page number,
% see the note about the +LP and -LP lines below. Comment out the +LP
% and uncomment the -LP.
%
% IF YOU WISH TO REMOVE PAGE NUMBERS, be sure that the includefoot line
% is uncommented and ALSO uncomment the \pagestyle{empty} a few lines
% below.
%

%% Use these lines for letter-sized paper
\usepackage[paper=letterpaper,
            %includefoot, % Uncomment to put page number above margin
            marginparwidth=1.2in,     % Length of section titles
            marginparsep=.05in,       % Space between titles and text
            margin=1in,               % 1 inch margins
            includemp]{geometry}

%% Use these lines for A4-sized paper
%\usepackage[paper=a4paper,
%            %includefoot, % Uncomment to put page number above margin
%            marginparwidth=30.5mm,    % Length of section titles
%            marginparsep=1.5mm,       % Space between titles and text
%            margin=25mm,              % 25mm margins
%            includemp]{geometry}

%% More layout: Get rid of indenting throughout entire document
\setlength{\parindent}{0in}

\usepackage[shortlabels]{enumitem}

%% Reference the last page in the page number
%
% NOTE: comment the +LP line and uncomment the -LP line to have page
%       numbers without the ``of ##'' last page reference)
%
% NOTE: uncomment the \pagestyle{empty} line to get rid of all page
%       numbers (make sure includefoot is commented out above)
%
\usepackage{fancyhdr,lastpage}
\pagestyle{fancy}
%\pagestyle{empty}      % Uncomment this to get rid of page numbers
\fancyhf{}\renewcommand{\headrulewidth}{0pt}
\fancyfootoffset{\marginparsep+\marginparwidth}
\newlength{\footpageshift}
\setlength{\footpageshift}
          {0.5\textwidth+0.5\marginparsep+0.5\marginparwidth-2in}
\lfoot{\hspace{\footpageshift}%
       \parbox{4in}{\, \hfill %
                    \arabic{page} of \protect\pageref*{LastPage} % +LP
%                    \arabic{page}                               % -LP
                    \hfill \,}}

% Finally, give us PDF bookmarks
\usepackage{color,hyperref}
\definecolor{darkblue}{rgb}{0.0,0.0,0.9}
\hypersetup{colorlinks,breaklinks,
            linkcolor=darkblue,urlcolor=darkblue,
            anchorcolor=darkblue,citecolor=darkblue}

%%%%%%%%%%%%%%%%%%%%%%%% End Document Setup %%%%%%%%%%%%%%%%%%%%%%%%%%%%
% =================================================================================================%

%%%%%%%%%%%%%%%%%%%%%%%%%%% Helper Commands %%%%%%%%%%%%%%%%%%%%%%%%%%%%

% The title (name) with a horizontal rule under it
% (optional argument typesets an object right-justified across from name
%  as well)
%
% Usage: \makeheading{name}
%        OR
%        \makeheading[right_object]{name}
%
% Place at top of document. It should be the first thing.
% If ``right_object'' is provided in the square-braced optional
% argument, it will be right justified on the same line as ``name'' at
% the top of the CV. For example:
%
%       \makeheading[\emph{Curriculum vitae}]{Your Name}
%
% will put an emphasized ``Curriculum vitae'' at the top of the document
% as a title. Likewise, a picture could be included:
%
%   \makeheading[\includegraphics[height=1.5in]{my_picutre}]{Your Name}
%
% the picture will be flush right across from the name.
\newcommand{\makeheading}[2][]%
        {\hspace*{-\marginparsep minus \marginparwidth}%
         \begin{minipage}[t]{\textwidth+\marginparwidth+\marginparsep}%
             {\large \bfseries #2 \hfill #1}\\[-0.15\baselineskip]%
                 \rule{\columnwidth}{1pt}%
         \end{minipage}}

% The section headings
%
% Usage: \section{section name}
\renewcommand{\section}[1]{\pagebreak[3]%
    \hyphenpenalty=10000%
    \vspace{1.3\baselineskip}%
    \phantomsection\addcontentsline{toc}{section}{#1}%
    \noindent\llap{\scshape\smash{\parbox[t]{\marginparwidth}{\raggedright #1}}}%
    \vspace{-\baselineskip}\par}

% An itemize-style list with lots of space between items
\newenvironment{outerlist}[1][\enskip\textbullet]%
        {\begin{itemize}[#1,leftmargin=*]}{\end{itemize}%
         \vspace{-.6\baselineskip}}

% An environment IDENTICAL to outerlist that has better pre-list spacing
% when used as the first thing in a \section
\newenvironment{lonelist}[1][\enskip\textbullet]%
        {\begin{list}{#1}{%
        \setlength{\partopsep}{0pt}%
        \setlength{\topsep}{0pt}}}
        {\end{list}\vspace{-.6\baselineskip}}

% An itemize-style list with little space between items
\newenvironment{innerlist}[1][\enskip\textbullet]%
        {\begin{itemize}[#1,leftmargin=*,parsep=0pt,itemsep=0pt,topsep=0pt,partopsep=0pt]}
        {\end{itemize}}

% An environment IDENTICAL to innerlist that has better pre-list spacing
% when used as the first thing in a \section
\newenvironment{loneinnerlist}[1][\enskip\textbullet]%
        {\begin{itemize}[#1,leftmargin=*,parsep=0pt,itemsep=0pt,topsep=0pt,partopsep=0pt]}
        {\end{itemize}\vspace{-.6\baselineskip}}

% To add some paragraph space between lines.
% This also tells LaTeX to preferably break a page on one of these gaps
% if there is a needed pagebreak nearby.
\newcommand{\blankline}{\quad\pagebreak[3]}
\newcommand{\halfblankline}{\quad\vspace{-0.5\baselineskip}\pagebreak[3]}

% Uses hyperref to link DOI
\newcommand\doilink[1]{\href{http://dx.doi.org/#1}{#1}}
\newcommand\doi[1]{doi:\doilink{#1}}

% For \url{SOME_URL}, links SOME_URL to the url SOME_URL
\providecommand*\url[1]{\href{#1}{#1}}
% Same as above, but pretty-prints SOME_URL in teletype fixed-width font
\renewcommand*\url[1]{\href{#1}{\texttt{#1}}}

% For \email{ADDRESS}, links ADDRESS to the url mailto:ADDRESS
\providecommand*\email[1]{\href{mailto:#1}{#1}}
% Same as above, but pretty-prints ADDRESS in teletype fixed-width font
%\renewcommand*\email[1]{\href{mailto:#1}{\texttt{#1}}}

%\providecommand\BibTeX{{\rm B\kern-.05em{\sc i\kern-.025em b}\kern-.08em
%    T\kern-.1667em\lower.7ex\hbox{E}\kern-.125emX}}
%\providecommand\BibTeX{{\rm B\kern-.05em{\sc i\kern-.025em b}\kern-.08em
%    \TeX}}
\providecommand\BibTeX{{B\kern-.05em{\sc i\kern-.025em b}\kern-.08em
    \TeX}}
\providecommand\Matlab{\textsc{Matlab}}




%%%%%%%%%%%%%%%%%%%%%%%% End Helper Commands %%%%%%%%%%%%%%%%%%%%%%%%%%%

%%%%%%%%%%%%%%%%%%%%%%%%% Begin CV Document %%%%%%%%%%%%%%%%%%%%%%%%%%%%

\begin{document}
\makeheading{Andrew J. Emerick\\ \textnormal{NSF Graduate Research Fellow - Astronomy}}

\section{Contact Information}

% NOTE: Mind where the & separators and \\ breaks are in the following
%       table.
%
% ALSO: \rcollength is the width of the right column of the table
%       (adjust it to your liking; default is 1.85in).
%
\newlength{\rcollength}\setlength{\rcollength}{1.4in}%
%
\begin{tabular}[t]{@{}p{\textwidth-\rcollength}p{\rcollength}}
%\href{http://www.cse.osu.edu/}%
%     {Department of Computer Science and Engineering} & \\
%\href{http://www.osu.edu/}{The Ohio State University}
3133 Broadway, Apt. 8  & 313-399-1179 \\
New York, NY 10025     & \email{emerick@astro.columbia.edu}\\
\end{tabular}

%\section{Objective}

%Insert text here if you want to
%\begin{innerlist}
%\item More information and auxiliary documents can be found at\\\url{http://www.tedpavlic.com/facjobsearch/}
%\end{innerlist}

\section{Research Interests}

Using high resolution computational simulations to study the formation and 
evolution of galaxies, from dwarf galaxies in our Local Group to massive galaxy clusters.
My current research, with Dr. Mordecai-Mark Mac Low and Dr. Jana Grcevich, 
focuses on using the \textsc{FLASH} astrophysics code to study the
evolution of satellite dwarf galaxies around our Milky Way.
With Dr. Greg Bryan and Dr. Mary Putman, I utilized \textsc{Enzo} to 
and \textit{yt} to produce synthetic absorption line observations of two massive,
simulated galaxy clusters. Previous
work has included magnetohydrodynamical simulations of astrophysical plasmas, and the
study of radio halos in galaxy clusters.

\section{Education}

\href{}{\textbf{Columbia University}},
\begin{outerlist}

\item[] Ph.D.,
		\href{}
			{Astronomy},
			\emph{Expected:} 2018/2019
\end{outerlist}

\vspace{.1in}
\href{http://www.umn.edu}{\textbf{University of Minnesota}},
Minneapolis, MN
\begin{outerlist}

%        \begin{innerlist}
%        \item Thesis Topic: \emph{asdf}
%        \item Advisors:
%              \href{}
%                   {name} and
%              \href{http://www.biostat.umn.edu/~sudiptob/}
%                   {Sudipto Banerjee, Ph.D}
%        \end{innerlist}

\item[] B.S.,
        \href{http://www.astro.umn.edu/}
             {Astrophysics},
             May 2013
        \begin{innerlist}
		\item Summa Cum Laude, with Distinction    
        \item Thesis Topic: \emph{Evolution of Weak Magnetic Fields in a Turbulent Plasma: A
Galaxy Cluster Context}
        \item Advisor:
              \href{http://homepages.spa.umn.edu/~twj/}
                   {Thomas W. Jones, Ph.D.}
        \end{innerlist}
        
\item[] B.S.,
        \href{http://www.physics.umn.edu/}
             {Physics},
             May 2013
        \begin{innerlist}
		\item Graduated with Distinction        
        \end{innerlist}
\end{outerlist}

% The 'iflong' 'fi' statements in the following are used to disable/enable
% descriptions of research experience when generating CV, depending on 
% need (see if statements at top of file)
\section{Research Experience}

\textbf{NSF Graduate Fellow} \hfill {Aug. 2014 - Present}

\textbf{Research Assistant} \hfill {Aug. 2013 - Present}
\begin{innerlist}

\item[] Department of Astronomy, Columbia University - New York, NY\\
        American Museum of Natural History - New York, NY
    Supervisors: Greg Bryan, Ph.D and Mordecai-Mark Mac Low, Ph.D
		Supervisors (previous): Mary Putman, Ph.D
		\begin{innerlist}
		\item Research utilizes large scale computational simulations to study
		galaxy formation and evolution on all scales, with a focus on dwarf galaxies.
		\item Currently implementing a chemodynamics method into the AMR, hydro code 
		\textsc{Enzo} using star-by-star modeling and including the effects of supernovae,
		stellar winds, cosmic rays, and full radiative transfer.
		\item Utilize following code projects: \href{http://enzo-project.org/}{Enzo}, 
		\href{http://yt-project.org/}{\textit{yt}}, \href{http://flash.uchicago.edu/site/}{FLASH}
		\end{innerlist}
\end{innerlist}

\iflong
\textbf{Undergraduate Research Assistant} \hfill {Dec. 2011 - Aug. 2013}
\begin{innerlist}

\item[] Minnesota Institute of Astrophysics, University of Minnesota - Minneapolis, MN\\
		Supervisors: Thomas W. Jones, Ph.D and David Porter, Ph.D
		\iflong
		\begin{innerlist}
		\item Studying evolution of weak magnetic fields in turbulent plasmas using
		ideal MHD simulations at the Minnesota Supercomputing Institute.
		\item A portion of this research constituted my senior undergraduate thesis.
		\end{innerlist}
		\fi
\end{innerlist}

\textbf{Undergraduate Research Assistant} \hfill {Jan. 2012 - May 2012}
\begin{innerlist}

\item[] Department of Physics, University of Minnesota - Minneapolis, MN\\
		Supervisor: Priscilla Cushman, Ph.D

		\begin{innerlist}		
		\item Worked with partner to characterize the gamma ray background in the 
		Cryogenic Dark Matter Search detector testing facility.
		\item Used high purity germanium detector to lay groundwork for construction
		of appropriate lead shielding for testing apparatus.
		\end{innerlist}

\end{innerlist}

\textbf{Undergraduate Research Assistant} \hfill {Oct. 2009 - Sep. 2011}
\begin{innerlist}

\item[] Minnesota Institute for Astrophysics, University of Minnesota - Minneapolis, MN\\
		Supervisor: Lawrence Rudnick, Ph.D

		\begin{innerlist}
		\item Worked on data processing and quality control for Green Bank Telescope
		observations done by graduate student
		\item Studied evolution of galaxy clusters using cluster radio halos as probes. Stacked
		faint, diffuse halos to independently confirm bi-modal distribution in radio halo properties.
		\end{innerlist}

\end{innerlist}

\textbf{Research Experience for Undergraduates} \hfill {Summer 2011}
\begin{innerlist}

\item[] Cyclotron Institute, Texas A\&M University - College Station, TX\\
        Supervisor: Ralf Rapp, Ph.D

        \begin{innerlist}
        \item Utilized analytical reproduction of lattice QCD results to motive revisit
        of problem of bottomonium binding scenarios in the QGP.
        \item Used new understandings to update bottomonium production code to predict 
        via observables, bottomonium yields at both RHIC and LHC.
        \end{innerlist}

\end{innerlist}

\textbf{Research Assistant} \hfill {Summer 2010}
\begin{innerlist}

\item[] Bonner Nuclear Laboratories, Rice University - Houston, TX\\
        Supervisor: Pablo Yepes, Ph.D

        \begin{innerlist}
        \item Utilized Monte Carlo and first principles code to simulate proton radiation
        therapy. 
        \item Improved first principles code with goal of more efficiently calculating relevant
        parameters in a fraction of Monte Carlo simulation run-times.
        \end{innerlist}

\end{innerlist}
\fi

\section{Refereed Journal Publications}
\vspace{-.1275in}
\begin{bibsection}
    \item \textbf{A. Emerick}, M-M. Mac Low, J. Grcevich, A. Gatto, \href{http://arxiv.org/abs/1605.02746}{``Gas Loss by Ram Pressure Stripping and Internal Feedback From Low Mass Milky Way Satellites''}, 2016, \textit{accepted for publication in ApJ}

    \item \textbf{A. Emerick}, G. Bryan, M. E. Putman,  \href{http://adsabs.harvard.edu/abs/2015MNRAS.453.4051E}{``Warm Gas in and Around Simulated Galaxy Clusters as Probed by Absorption Lines''}, 2015 \textit{MNRAS} \textbf{453}
    
    \item \textbf{A. Emerick}, X. Zhao, R. Rapp, \href{http://adsabs.harvard.edu/abs/2012EPJA...48...72E}{``Bottomonia in the Quark-Gluon Plasma and
	their Production at RHIC and LHC''}, \textit{Eur. Phys. J. A} (2012)
	\textbf{47}:72	

	
	\item S. Brown, \textbf{A. Emerick}, L. Rudnick, G. Brunetti, \href{http://adsabs.harvard.edu/abs/2011ApJ...740L..28B}{``Probing the Off-State of
	Cluster Radio Halos''}, 2011 \textit{ApJ} \textbf{740} L28 
	
\end{bibsection}

\section{Talks}
\vspace{-.125in}
\begin{bibsection}
    \item \textbf{A. Emerick}, M. Putman, G. Bryan, "Warm Gas in and Around Simulated
    Galaxy Clusters as Probed by Absorption Line Studies", The Role of HI in Galxies,
    Kuching, Malaysia (Fall 2014)
\end{bibsection}

\section{Conference Publications}
\vspace{-.125in}
\begin{bibsection}
  \item \textbf{A. Emerick}, M-M. Mac Low, J. Grcevich, A. Gatto, "The Surprising Inefficiency of Ram Pressure Stripping and Feedback in Quenching the Lowest Mass Milky Way Satellites". 
 `` Mapping the Pathways of Galaxy Transformation Across Time and Space'', Avalon, Californa,
    August 2016
  
  \item \textbf{A. Emerick}, M-M. Mac Low, J. Grcevich, A. Gatto, "The Evolution of Satellite
  Dwarf Galaxies via Stripping and Supernova Feedback in a Milky Way Type Halo". 
  Lorentz Center Workshop, "The Life and Death of Satellite Galaxies". Leiden, Netherlands 2015

	\item X. Zhao, \textbf{A. Emerick}, R. Rapp, "In-Medium Quarkonia at SPS, RHIC, and LHC"
	\textit{Nuclear Physics A, Vol. 904, p. 611-614c}. Quark Matter 2012 - Proceedings. 
	\href{http://www.sciencedirect.com/science/article/pii/S0375947413002042}{Abstract}
	
	\item \textbf{A. Emerick}, T.W. Jones, D. Porter, "Simulation of Turbulence and Magnetic
	Field Evolution in Astrophysical Plasmas", \textit{UMN Digital Conservatory:
	Undergraduate Research Presentations}. \href{http://purl.umn.edu/123023}{Poster}

	\item \textbf{A. Emerick}, X. Zhao, R. Rapp, "Bottomonium in the QGP: production at
	RHIC and LHC." \textit{Fall Meeting of the APS Division of Nuclear Physics: Bulletin
	of the American Physical Society, Volume 56, Number 12}.
	\href{http://meeting.aps.org/Meeting/DNP11/Event/156070}{Poster abstract}

	\item \textbf{A. Emerick}, 	S. Brown, L. Rudnick, "Stacking Detection of Diffuse Radio
	Halo Emission in Galaxy Clusters". In: \textit{AAS Meeting \# 218, \# 408.26; Bulletin of 
	the American Astronomical Society, Vol. 43, 201}. 
	\href{http://adsabs.harvard.edu/abs/2011AAS...21840826E}{Poster abstract}.
	
	\item \textbf{A. Emerick}, S. Brown, L. Rudnick, "Examination of Radio Halos and 
	Corresponding X-ray Emission in Galaxy Clusters", \textit{UMN Digital Conservatory:
	Undergraduate Research Presentations}.
	\href{http://conservancy.umn.edu/handle/104933}{Poster}.
\end{bibsection}

\section{Unpublished Works}
\vspace{-.125in}
\begin{bibsection}
	\item \textbf{A. Emerick}, "Evolution of Weak Magnetic Fields in a Turbulent Plasma: A 
	Galaxy Cluster Context", \textit{Submitted to the University Honors Program at the 
	University of Minnesota in partial fullfillment of the requirements for the degree of
	Bachelor of Science summa cum laude in Astrophysics}. 
	\href{http://purl.umn.edu/155303}{Full paper}
\end{bibsection}

\section{Teaching Experience}
\begin{innerlist}
\item Lab T.A. Astronomy W1904: Astronomy Lab II \hfill {Spring 2015}
\item Lab T.A. Astronomy C1904: Astronomy Lab I \hfill {Fall 2014}
\item T.A. Astronomy C1403: Earth, Moon, and Planets \hfill {Spring 2014}
\item T.A. Astronomy C1836: Stars and Atoms \hfill {Fall 2013}
\end{innerlist}

\halfblankline

\section{Public Outreach}
\textbf {Writer for Astrobites} \hfill {Nov. 2013 - Present} \\
\begin{innerlist}
    \iflong
    \item \href{http://astrobites.com/}{Astrobites} is a blog type website
    geared towards daily summaries of astronomy journal papers, readable by
    an undergraduate astronomy/physics student.
    \fi	  
\end{innerlist}

\halfblankline
% Add a little space to nudge next ``Conference Publications'' marginpar
% down to make room for tall ``Submitted Journal Publications''
% marginpar. If there are enough submitted journal publications, this
% space will not be needed (and should be removed).
%\vspace{0.1in}

%\section{Papers in Preparation}
%\vspace{-.1in}
%\begin{bibsection}
%    \item Toomey, T.L., Erickson, D.J., Carlin, B.P., Lenk, K.M., {\bf Quick, H.S.}, and Harwood, E.M. ``Do neighborhood attributes moderate the relationship between alcohol establishment density and crime?"
%    \item {\bf Quick, H.}, Banerjee, S., and Carlin, B.P. ``Heteroscedastic variances in areally referenced temporal processes with an application to California asthma hospitalization data.''

%    \item {\bf Quick, H.}, Carlin, B.P., and Banerjee, S. ``Space-time Gaussian process modeling of temporal air pollution gradients."
%\end{bibsection}

\section{Awards}
Graduate Awards
\begin{innerlist}
\item NSF Graduate Research Fellow \hfill Fall 2014, 5 yr\\
\item Dean's Fellowship \hfill Fall 2013, 5 yr \\
	  Department of Astronomy, Columbia University
\end{innerlist}

\halfblankline

Travel Awards
\begin{innerlist}
\item Conference Experience for Undergraduates\hfill Oct. 2011\\
APS DNP Fall Meeting
\item Funding from University of Minnesota 
      and APS Minority Scholarship \hfill May 2011 \\
      AAS 218$^{th}$ Meeting  - Summer Boston
\end{innerlist}

\halfblankline

Research Grants
\begin{innerlist}
\item XSEDE Computing Startup Grant on the Stampede Cluster \hfill 2015-2016 \\
\item Undergraduate Research Opportunities Program Grant \hfill Spring 2012 \\
University of Minnesota
\item Undergraduate Research Opportunities Program Grant \hfill Fall 2010 \\
University of Minnesota 
\end{innerlist}

\halfblankline

Undergraduate Scholarships
\begin{innerlist}
\item J. Morris Blair Scholarship in Physics \hfill 2012-2013\\
Dept. of Physics (UMN)
\item Laverne and Ted Jones Undergraduate Scholarship \hfill 2012-2013 \\
Minnesota Institute for Astrophysics (UMN) 
\item Astronaut Scholarship Foundation Scholarship \hfill 2012-2013
\item Minnesota Space Grant Consortium Scholarship \hfill Spring 2011 and 2012
\item Franklin Scholarship \hfill Fall 2011\\
School of Physics and Astronomy (UMN)
\item American Physical Society Minority Scholarship \hfill 2010-2012
\item Gold National Scholarship \hfill 2009-2013\\
University of Minnesota
\end{innerlist}

%
%\section{Presentations}
%Statistical Meetings
%\begin{innerlist}
%\item Workshop on Environmetrics, Raleigh, NC \hfill Oct 2012
%\item Joint Statistical Meetings, San Diego, CA \hfill Aug 2012
%\item Biometric Society (ENAR) Regional Meeting, Washington, D.C. \hfill Apr 2012
%\item Case Studies in Bayesian Statistics and\hfill Oct 2011\\
%Machine Learning, Pittsburgh, PA
%\item Biometric Society (ENAR) Regional Meeting, Miami, FL \hfill Mar 2011
%\item IMS/ISBA Joint International Meeting, Park City, UT \hfill Jan 2011
%\end{innerlist}
%
%\halfblankline
%
%University of Minnesota
%\begin{innerlist}
%\item Mostly Markov Chain Seminar Series \hfill Nov 2011
%\item School of Public Health Research Day \hfill Apr 2011
%\end{innerlist}
%
%
%\section{Teaching Experience}
%
%\textbf{Teaching Assistant} \hfill {Springs 2011--12}
%Co-instructor \hfill {Summer 2013}
%\begin{innerlist}
%\item[] PUBH 6400 - Topics in Hierarchical Bayesian Analysis\\
%        with Bradley P. Carlin\\
%        Division of Biostatistics,\\
%        University of Minnesota
%\end{innerlist}
%
%Teaching Assistant \hfill {Springs 2011--12}
%\begin{innerlist}
%
%\item[] PUBH 7440 - Introduction to Bayesian Analysis\\
%        Instructor: Bradley P. Carlin, Ph.D\\
%        Division of Biostatistics,\\
%        University of Minnesota
%\end{innerlist}
%
%\section{Service}
%Recruiting Committee, Division of Biostatistics \hfill {May 2010 -- Present}
%\begin{innerlist}
%    \item Assist with planning of annual Division of Biostatistics Open House and Admitted Student Visit Days
%    \item Meet with prospective and admitted students %; answer questions from a student's perspective
%\end{innerlist}
%
%\halfblankline
%
%Student Member of Search Committee for the \hfill {June 2010 -- Aug 2010}\\
%SPH Coordinator of Recruitment and Student Leadership
%\begin{innerlist}
%    \item Assisted in job search for the SPH Coordinator of Recruitment and Student Leadership
%    \item Reviewed applications, conducted interviews
%\end{innerlist}

\section{Organizations}
\begin{innerlist}
\item American Physical Society (APS); Society of Physics Students (SPS), Sigma Pi
Sigma: Physics Honors Society
\end{innerlist}


\section{Skills}
Computer Programming:
\begin{innerlist}
    \item C$++$, Fortran, Python, LaTex, R, UNIX shell scripting, 
    GNU make
\end{innerlist}

%\halfblankline

\section{References}
Greg Bryan
\begin{innerlist}
\item[] Professor of Astronomy \hfill {Phone: 1-212-854-6837}\\
Department of Astronomy \hfill{E-mail: gbryan@astro.columbia.edu }\\
Columbia University
\end{innerlist}

\halfblankline

Thomas W. Jones
\begin{innerlist}
\item[] Professor of Astronomy \hfill {Phone: 612-624-1699}\\
Minnesota Institute for Astrophysics \hfill{E-mail: twj@astro.umn.edu}\\
University of Minnesota
\end{innerlist}

\halfblankline

Lawrence Rudnick
\begin{innerlist}
\item[] Distinguished Teaching Professor \hfill {Phone: 612-624-3396}\\
Minnesota Institute for Astrophysics \hfill{E-mail: larry@astro.umn.edu}\\
University of Minnesota
\end{innerlist}

\end{document}
